%\documentclass[handout]{beamer}
\documentclass[evncountsect]{beamer}
\usepackage{amsmath,algorithm,algorithmic,graphicx,amsfonts,amsthm,color,pgf,tikz,wrapfig,amsfonts,multicol,wasysym,animate}
\usepackage[absolute,overlay]{textpos}
\usetikzlibrary{arrows}
\beamertemplatenavigationsymbolsempty
\setbeamercovered{transparent = 0}
\useoutertheme[subsection=false]{miniframes}
\bibliographystyle{apalike}
\setlength{\fboxrule}{1pt}

\title{RNA-seq and Adaptive Shrinkage}
\author{Mengyin Lu and David Gerard}
\institute[UChicago]{
  Department of Human Genetics\\
  University of Chicago\\
  Boss: Matthew Stephens
}
\date[September 2016]{September, 2016}

% available at https://github.com/matze/mtheme
\usetheme{metropolis}

\begin{document}


\begin{frame}
  \titlepage
\end{frame}

\begin{frame}{RNA-seq}
  What is RNA-seq?
  \begin{enumerate}
  \item Each cell in our body contains a population of RNA fragments.
  \item RNA-seq uses next-gen sequencing to measure the relative expression of genes.
  \item Relative differences in the concentrations of these RNA fragments between groups tells us interesting things. 
    \begin{enumerate}
    \item Which genes are influenced by a drug?
    \item How do cancer patients differ from non-cancer patients?
    \item Which genes are important in liver tissue versus muscle tissue?
    \end{enumerate}
  \item Many improvements over older technology.
  \end{enumerate}

  % Improvements over older technology:
  % \begin{enumerate}
  % \item Do not need prior knowledge of a gene to measure its expression level.
  % \item Can compare genes within a sample.
  % \item More accurate measurements of low/high expressed genes.
  % \end{enumerate}
\end{frame}


\begin{frame}{Many Problems}
  \begin{enumerate}
  \item Count Data: A new type of data that requires new normalization
    procedures/pipelines.
  \item Small Sample Sizes: Low power. Hard to detect differences between groups.
  \item Unobserved Variables: Can ruin analyses. Can make results uninterpretable.
    \begin{enumerate}
    \item Which lab/technician processed a sample?
    \item The ancestry of the sample.
    \item Subject attributes such as age or sex.
    \end{enumerate}
  \end{enumerate}
\end{frame}

\begin{frame}{Our Work}
  \begin{enumerate}
  \item Count Data: Develop optimal pipelines (VOOM-LIMMA)
  \item Small Sample Sizes:
    \begin{enumerate}
    \item Harness the power of ASH.
    \item Borrow strength between variables.
    \end{enumerate}
  \item Unobserved Variables:
    \begin{enumerate}
    \item Integrate ``factor-augmented regression'' models with the shrinkage ideas from ASH.
    \item Modify known approaches to optimize the performance of ASH.
    \end{enumerate}
  \end{enumerate}
\end{frame}

\end{document}